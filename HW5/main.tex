\documentclass{article}

\usepackage[utf8]{inputenc} 
\usepackage[russian]{babel}
\usepackage{amsmath,amssymb}
\usepackage{tipa}
\usepackage{hyperref}
\usepackage{graphicx}
\usepackage{comment}
\binoppenalty=10000
\relpenalty=10000
\title{Алгоритмы и Структуры Данных ДЗ-5}
\date{26.10.2019}
\usepackage{multicol}

\author{Гарипов Роман М3138}

\begin{document}
  \pagenumbering{gobble} 
  \maketitle
  \newpage
  \pagenumbering{arabic}

\section*{Задача \textnumero 1}
Покажем, что если для сортирующей сети не существует компараторов для двух соседних позиций $i$ и $i + 1$, то это вовсе не сортирующая сеть.
Подадим на вход нашей сортирующей сети такую перестановку, что она отсортирована, за исключением того, что два подряд идущих элемента $i$ и $i + 1$ образуют инверсию. Для ясности, приведём пример с $i = 3$. $p = [1, 2, 4, 3 \dots n - 1, n]$. Понятно, что такая перестановка не будет отсортирована. Следовательно, наша сеть не является сортирующей, так как есть входные данные, которые сеть не отсортирует.

\section*{Задача \textnumero 2}
Будем строить нашу сортирующую сеть по слоям.
\newline
 На первом слое сравним наш элемент(обозначим его вход с номером $n$, остальные входы имеют номера $1 \dots n - 1$) с каким-то одним элементом(поставим компаратор). Получим, что множество позиций на которых может находиться элемент со входа $n$ теперь состоит из двух позиций.   \newline
 На втором слое сравним каждую из позиций нашего множества, с какой-нибудь одной позицией, которая ещё не находится в нашем множестве. И продолжим делать так, пока не получим множество из $n$ элементов.
\newline 
Получим, что на каждой слое размер множества увеличивается в $2$ раза, за исключением, возможно, последнего перехода, там может не хватить элементов для сравнения, но от этого глубина не измениться, ведь все равно на этой глубине есть какие-то компараторы. 
\newline
 Важно, что мы не можем добавить более одного компаратора для какой-то позиции на каждом слое, ведь если мы захотим добавить для какого-то элемента второй компаратор, его придётся разметить на третьем слое, это будет неоптимально.  
\newline 
Отсюда становится понятно, что за наращение одного слоя не получится увеличить наше множество больше чем в два раза.
\newline
Получаем, что всего слоёв будет хотя бы $\log_{2}(n)$.
\end{document}