\documentclass{article}

\usepackage[utf8]{inputenc} 
\usepackage[russian]{babel}
\usepackage{amsmath,amssymb}
\usepackage{tipa}
\usepackage{hyperref}
\usepackage{graphicx}
\usepackage{comment}
\binoppenalty=10000
\relpenalty=10000
\title{Алгоритмы и Структуры Данных ДЗ-10}
\date{01.12.2019}
\usepackage{multicol}

\author{Гарипов Роман М3138}

\begin{document}
  \pagenumbering{gobble} 
  \maketitle
  \newpage
  \pagenumbering{arabic}

\section*{Задача \textnumero 1}
\subsection*{Считаем ответ на задачу, используя вспомогательную динамику}
Будем считать две динамики, первая - ответ на задачу, по множеству будем получать гамильтонов цикл минимального веса - $dp_{0}[X]$. 
\newline
Вторая - вспомогательная, будет выглядеть так :
\newline
Пусть мы зафиксировали какую-то вершину $s$. Множество $X$, которое будет использовано далее не может содержать вершины с элементами больше $s$.
\newline
$dp_{1}[X][i]$ - гамильтонов путь минимального веса, начинающийся в зафиксированной вершине $s$ и заканчивающийся в вершине $i$, проходящий по всем вершинам из $X$.
\newline
Когда посчитали такую динамику для фиксированного $s$, можем попробовать улучшить $dp_{0}$ для всех множеств, содержащих вершины с номерами не больше $s$. Для этого просто выберем какое-то множество $X$, какой-то конец пути $i$ и если $(i \to s) \in E$, тогда $dp_{0}[X] = \min(dp_{0}[X], dp_{1}[X][i] + w_{(i \to s)})$. Таким образом, мы прошли по пути от $s$ до $i$ и потом по ребру $(i \to s)$, получили цикл. 
\newline
Если перебрать все $s$, то посчитаем для всех подмножеств гамильтонов цикл максимального веса.
\subsection*{Считаем вспомогательную динамику}
Пусть зафиксировали $s$. 
\newline
Положим $dp_{1}[\{s\}][s]$ = 0 - база динамики. 
\newline
Пусть имеем уже посчитаное состояние $dp[X][i]$. Тогда, переберём вершину $j$ : $$j \in [1 \dots s], j \notin X, (i \to j) \in E$$ 
$$dp_{1}[X \cup{} \{j\}][j] = \min(dp_{1}[X \cup \{j\}][j], dp_{1}[X][i] + w_{(i \to j)})$$.
\newline
У нас был путь из вершины $s$ в $i$, мы прошли по ребру $(i \to j)$ и попытались улучшить ответ для нового состояния. Поскольку мы переберём все возможные ребра из $i$, то получим все возможные варианты и выберем из них минимум. Поэтому наша динамика посчитается верно.
\newline
После того как посчитали все состояния динамики,  попытаемся улучшить ответ для основной динамики как это было описано ранее.
\subsection*{Ассимптотика}
Решение работает за $$\sum_{i = 1}^{n}(i^{2}\cdot2^{i}) = \mathcal{O}(2^{n}n^{2})$$
\newline
По индукции :  $$\sum_{i = 1}^{n}(i^{2}2^{i}) \leq 2^{n + 1}n^{2}$$.
\section*{Задача \textnumero 4}
Для того чтобы решить эту задачу, научимся считать количество гамильтоновых путей с фиксированным началом и концом в любом подномжестве вершин, а потом поймем как из этого получить количество простых циклов.
\newline
\subsection*{Динамика}
$dp[S][i] = $ количество гамильтоновых путей в множестве вершин $S \subset V$, которые начинаются в $\min(S)$ - вершина с минимальным номером, и заканчиваются в вершине $i$.
\subsection*{База динамики}
Положим $\forall i \in [1\dots n], X = \{i\}$ : $dp[X][i] = 1$ 
Для всех одноэлементных множеств сказали, что в них ровно $1$ гамильтонов цикл который начинается в $i$ и заканчивается там же.
Все остальные состояния проинициализируем нулями.
\subsection*{Динамический переход}
Хотим посчитать $dp[Y][i]$ для каких-то конкретно выбранных $Y$ и $i$, причем важно, что $i \neq \min(Y)$. Пусть для всех множеств размером меньше $|Y|$ уже посчитали динамику.
Тогда будем пересчитывать текущее состояние так :
$$dp[Y][i] = \sum_{j \in [1\dots n], (j \to i) \in E}(dp[Y \setminus \{i\}][j])$$
Мы взяли все пути, которые начинались в $\min(Y)$ и заканчивались в $j$ и добавили ребро. Посколько мы сделали это для всех $j$, это и будет количество гамильтоновых путей, начинающихся в $\min(Y)$ и заканчивающихся в вершине $i$.
\newline
Требование на то,что $i \neq \min(Y)$ необходимо для того, чтобы путь начинался строго в $\min(Y)$, так как простые циклы равны с точностью до циклических сдвигов, мы не хотим посчитать лишние.
\newline
Вся динамика считается за $\mathcal{O}(2^{n}n^{2})$, так как мы перебираем сначала подмножество, а потом две вершины.
\subsection*{Ответ}
Теперь осталось понять как по этой динамике вычислить ответ на задачу.
Ответом будет следующая величина : $$\frac{1}{2}\sum_{|X| \geq 2, i \in [1\dots n], (i \to \min(X)) \in E}(dp[X][i])$$
Мы взяли гамильтонов путь длины хотя бы $2$, который начинался в $\min(X)$ и заканчивался в $i$, причем потребовали, чтобы было ребро $(i \to \min(X))$. Пройдемся по этому пути, а потом по ребру и получим цикл. Мы выбрали все возможные подмножества вершин и все возможные концы путей, следовательно посчитали все простые циклы, но каждый посчитали $2$ раза, потому что, к примеру a-b-c и a-c-b это разные пути, но один и тот же цикл.
\end{document}