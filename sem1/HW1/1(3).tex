\documentclass{article}

\usepackage[utf8]{inputenc} % если ваш файл содержит русский текст, нужно указать кодировку
\usepackage[russian]{babel} % для того, чтобы писать русский текст
\usepackage{amsmath,amssymb}
\usepackage{tipa}
\usepackage{hyperref} % для вставки ссылок
\usepackage{graphicx}
\binoppenalty=10000
\relpenalty=10000
\title{Homework 1}
\date{10.09.2019}
\author{Roman Garipov M3138}
\usepackage{comment}

% до этого места была преамбула, тут можно задавать разные значения переменных, включать пакеты, а также указывать вещи, которые не суждено поменять посреди документа
\begin{document}
  \pagenumbering{gobble} % сейчас будет титульная страницу, отключим нумерацию страниц$\mathcal{O}}
  \maketitle % эта команда печатает титульную страницу
  \newpage % эта команда начинает новую страницу
  \pagenumbering{arabic} % включим нумерацию страниц обратно

\section{Задача \textnumero 1}
Докажите по определению, что 	$ \frac{n^{3}}{6} - 7n^{2} = \Omega(n^{3})$.
\newline
   $ \frac{n^{3}}{6} - 7n^{2} = \Omega(n^{3})$ значит, что $ \exists $ $c > 0,  n \in \mathbb{N}: \forall$ $n_{0} > n, f(n_{0}) \geqslant cg(n_{0})$. 
\newline
Чтобы показать, что это правда, достаточно найти такие $c$ и $n$, что выполняется определение. 
\newline	
Выберем $c = \frac{1}{36}$
\newline   
Тогда получим : $$ \frac{n^{3}}{6} - 7n^{2} \geqslant \frac{n^{3}}{36} $$
$$n^{3} - 42n^{2} \geqslant \frac{n^{3}}{6}$$
$$\frac{5}{6}n^{3} - 42n^{3} \geqslant 0$$
$$5n^{3} - 252n^{2} \geqslant 0$$
$$n^{2}(5n - 252) \geqslant 0$$ 
$$5n - 252 \geqslant 0$$
$$n \geqslant \frac{252}{5}$$ 
$$n \geqslant 50.4$$
$$n \geqslant 51$$
Мы нашли такие $c$ и $n$, что определение выполняется, следовательно доказали равенство.
      
\section{Задача \textnumero 2}


\section{Задача \textnumero 3}
Докажите или опровергните, что $\log(n!) = \Theta(n\log(n))$.
\newline
По определению $f(n) = \Theta(g(n))$ значит, что :
\begin{center}
$\exists  c_{1} > 0, c_{2} > 0, N \in \mathbb{N} : \forall n > N$,  $c_{1}g(n) \leq f(n) \leq c_{2}g(n)$
\end{center}
\begin{center}
Рассмотрим левое неравенство и запишем его в следующем виде:
\end{center}
$$c_{1}n\log(n) \leq \log(n!)$$
\begin{center}
Преобразуем
$\log_{a}(b) = \frac{\log_{c}(b)}{\log_{c}(a)}$, получаем $\log_{a}(b)\log_{c}(a) = \log_{c}(b)$

$\log_{2}(n!) = \log_{2}(n)\log_{n}(n!)$
\newline
$c_{1}n\log_{2}(n) \leq \log_{2}(n)\log_{n}(n!)$
\newline
$c_{1}n \leq \log_{n}(n!)$
\newline
$n^{c_{1}n} \leq n^{\log_{n}(n!)}$
\newline
$n^{c_{1}n} \leq n!$
\end{center}


Для $c > 1$ это всегда неверно.
\newline
Если решать это неравенство для какого-то конкретного $  0 < c < 1$, то мы получим ограничение сверху на $n$, так как всегда найдется $n$ такое, что : $n^{c_{1}n} \geq n!$. Следовательно, это не будет выполнятся для какого-то аргумента, начиная с какого-то $N$. Значит, $\log(n!) \neq \Theta(n\log(n))$
\newpage

\section{Задача \textnumero 4}

Найти $\mathcal{O}$  и  $\Omega$ для $T(n) = 2T(\frac{n}{2}) + \frac{n}{\log(n)}$.
\subsection{$\mathcal{O}$}
Я утверждаю, что $T(n) = \mathcal{O}(n\log(n))$. Докажем по индукции. База ~--- $n = 2, T(2) = 2$
\newline
Предположение индукции : Для всех $n_{0} < n$, $T(n_{0}) \leq cn_{0}\log(n_{0})$. Докажем, что и для $n$ это утверждение верно.
\newline
$T(n) = 2T(\frac{n}{2}) + \frac{n}{\log(n)}$ $\leq$ $2c\frac{n}{2}\log(\frac{n}{2}) + \frac{n}{\log(n)}$  (по предположению индукции)
\newline
$T(n) = 2T(\frac{n}{2}) + \frac{n}{\log(n)}$ $\leq$ $cn(\log(n) - 1) + \frac{n}{\log(n)} = cn\log(n) - cn + \frac{n}{\log(n)}$ .
\newline
$cn\log(n) - cn + \frac{n}{\log(n)}$ $\leq$ $cn\log(n)$ (Домножим обе части неравенства на $\log(n)$)
$cn\log^{2}(n) - cn\log(n) + n$ $\leq$ $cn\log^{2}(n)$.
\newline
$n$ $\leq$ $cn\log(n)$. Например, При $c$ $= 1$ получаем верное неравенство. %Получили верное неравенство, следовательно доказали, что $T(n)$ $\leq$ $cn\log(n)$.%

\subsection{$\Omega$}
Докажем, что $T(n) = 2T(\frac{n}{2}) + \frac{n}{\log(n)} = \Omega(n)$
\newline
А именно, что $T(n) = 2T(\frac{n}{2}) + \frac{n}{\log(n)}$ $\geq$ $cn$.
\newline
База индукции : $n = 2$, $T(2) = 2$ 
\newline
Предположение индукции : $T(n) = 2T(\frac{n}{2}) + \frac{n}{\log(n)}$ $\geq$ $cn$
%Предположим, что для всех чисел меньших $n$ это верно.
\newline%
$T(n) = 2T(\frac{n}{2}) + \frac{n}{\log(n)}$ $\geq$ $2\frac{cn}{2} + \frac{n}{\log(n)}$ $\geq$ $cn$
\newline
$cn + \frac{n}{\log(n)}$ $\geq$ $cn$. Например, при $c = 1$ это правда.
\newline
Что и требовалось доказать.
\end{document}