\documentclass{article}

\usepackage[utf8]{inputenc} % если ваш файл содержит русский текст, нужно указать кодировку
\usepackage[russian]{babel} % для того, чтобы писать русский текст
\usepackage{amsmath,amssymb}
\usepackage{tipa}
\usepackage{hyperref} % для вставки ссылок
\usepackage{graphicx}
\binoppenalty=10000
\relpenalty=10000
\title{Homework 1}
\date{22.09.2019}
\usepackage{multicol}

\author{Roman Garipov M3138}

\begin{document}
  \pagenumbering{gobble} 
  \maketitle
  \newpage
  \pagenumbering{arabic}

\section{Задача \textnumero 1}
Докажите по определению, что 	$ \frac{n^{3}}{6} - 7n^{2} = \Omega(n^{3})$.
$$$$


   $ \frac{n^{3}}{6} - 7n^{2} = \Omega(n^{3})$ значит, что $ \exists $ $c > 0,  n \in \mathbb{N}: \forall$ $n_{0} > n, f(n_{0}) \geqslant cg(n_{0})$. 
   Чтобы показать, что это правда, достаточно найти такие $c$ и $n$, что выполняется определение. 
\begin{center}
	Выберем $c = \frac{1}{36}$. 
   
   Тогда получим : $ \frac{n^{3}}{6} - 7n^{2} \geqslant \frac{n^{3}}{36} $
   
    $n^{3} - 42n^{2} \geqslant \frac{n^{3}}{6}$; 
    
    $\frac{5}{6}n^{3} - 42n^{3} \geqslant 0$;
    
     $5n^{3} - 252n^{2} \geqslant 0$; 
     
      $n^{2}(5n - 252) \geqslant 0$ 
    
     $5n - 252 \geqslant 0$; 
     
     $n \geqslant \frac{252}{5}$; $ n \geqslant 50.4$; 
     
     $n \geqslant 51$.
\end{center}     
      Мы нашли такие $c$ и $n$, что определение выполняется, следовательно доказали равенство.
      
\section{Задача \textnumero 2}
\begin{multicols}{3}
\begin{enumerate}
\item $1$
\item $n^{\frac{1}{\log(n)}}$
\item $(\frac{3}{2})^{2}$
\item $\log(\log(n))$
\item $\sqrt{\log(n)}$
\item $\sqrt{2}^{\log(n)}$
\item $\log^{2}(n)$
\item $n$
\item $2^{\log(n)}$
\item $\log(n!)$
\item $n\log(n)$
\item $\log(n)!$
\item $n^{2}$
\item $4^{\log(n)}$
\item $n^{3}$
\item $\log^{\log(n)}(n)$
\item $n^{\log(\log(n))}$
\item $n2^{n}$
\item $e^{n}$
\item $n!$
\item $(n + 1)!$
\item $2^{2^{n}}$
\item $2^{2^{n + 1}}$
\end{enumerate}
\end{multicols}

\section{Задача \textnumero 3}
Докажите или опровергните, что $\log(n!) = \Theta(n\log(n))$.
\newline
По определению $f(n) = \Theta(g(n))$ значит, что :
\begin{center}
	$\exists$   $c_{1} > 0, c_{2} > 0, N \in \mathbb{N} : \forall n > N$,  $c_{1}g(n) \leq f(n) \leq c_{2}g(n)$
\end{center}
	Рассмотрим левое неравенство и запишем его в следующем виде:
$$c_{1}n\log(n) \leq \log(n!)$$

Преобразуем	$\log_{a}(b) = \frac{\log_{c}(b)}{\log_{c}(a)}$, получаем
\begin{center} $\log_{a}(b)\log_{c}(a) = \log_{c}(b)$
	
$\log_{2}(n!) = \log_{2}(n)\log_{n}(n!)$

$c_{1}n\log_{2}(n) \leq \log_{2}(n)\log_{n}(n!)$

$c_{1}n \leq \log_{n}(n!)$

$n^{c_{1}n} \leq n^{\log_{n}(n!)}$

$n^{c_{1}n} \leq n!$
\end{center}
Аналогично для правого неравенста, получаем :
	$$n^{c_{1}n} \leq n! \leq n^{c_{2}n}$$
Если получится подобрать такие константы $c_{1}$ и $c_{2}$, что неравенство выполняется, мы докажем, что $\log(n!) = \Theta(n\log(n))$

\subsection{Выбираем $c_{1}$}
Попробуем взять $c_{1} = \frac{1}{4}$.  Получаем : $n^{\frac{n}{4}} \leq n!$ . 
Для того, чтобы показать, что это верно, оценим $n!$ .  
$$n! = 1 \cdot 2 \cdot 3 \dots (n - 2) \cdot (n - 1) \cdot n$$

Посмотрим на группы элементов выбранных следующим образом:
	\begin{itemize}
		\item $(1 \cdot 2 \cdot (n - 1) \cdot(n - 2)) \geq n$ 
		\item $(3 \cdot 4 \cdot (n - 4) \cdot (n - 3)) \geq n$ 
		\item $(5 \cdot 6 \cdot (n - 6) \cdot (n - 5)) \geq n$
		\newline
		\vdots
		\item $((\frac{n}{2} - 1) \cdot (\frac{n}{2}) \cdot (\frac{n}{2} + 1) \cdot (\frac{n}{2} + 2)) \geq n$
	\end{itemize}
	(Таким образом будут сгруппированы элементы, если $n$ кратно $4$, в другом случае будет аналогично, только в последнем неравенстве элементы будут начинаться с $\frac{n}{2}$)
\newline
Всего таких неравенств $\frac{n}{4}$, перемножим их и получим следующее :
$$ n! \geq n^{\frac{n}{4}}$$
\subsection{Выбираем $c_{2}$}
Попробуем взять его равным $1$.  Тогда, правая сторона неравенства будет выглядеть так : 
$$n! \leq n^{n}$$
$$n! = \underbrace{1 \cdot 2 \cdot 3 \dots (n - 2) \cdot (n - 1) \cdot n}_{\mbox{n членов}}$$
$$n^{n} = \underbrace{n \cdot n \cdot n \dots n \cdot n \cdot n}_{\mbox{n членов}}$$
\newline

Отсюда видно, что $n! \leq n^{n}$ начиная, к примеру, с $n \geq 1$.
\subsection{Итого}
Мы нашли такие $c_{1}$ и $c_{2}$, что:   $$c_{1}n\log(n) \leq \log(n!) \leq c_{2} n\log(n)$$ Получается, что $\log(n!) = \Theta(n\log(n))$.
\section{Задача \textnumero 4}

Найти $\mathcal{O}$  и  $\Omega$ для $T(n) = 2T(\frac{n}{2}) + \frac{n}{\log(n)}$.
\subsection{$\mathcal{O}$}
Я утверждаю, что $T(n) = \mathcal{O}(n\log(n))$. Докажем по индукции. База ~--- $n = 2, T(2) = 2$, 
\newline
Предположение индукции : Для всех $n_{0} < n$, $T(n_{0}) \leq cn_{0}\log(n_{0})$. Докажем, что и для $n$ это утверждение верно.
\newline
$T(n) = 2T(\frac{n}{2}) + \frac{n}{\log(n)}$ $\leq$ $2c\frac{n}{2}\log(\frac{n}{2}) + \frac{n}{\log(n)}$  (по предположению индукции)
\newline
$T(n) = 2T(\frac{n}{2}) + \frac{n}{\log(n)}$ $\leq$ $cn(\log(n) - 1) + \frac{n}{\log(n)} = cn\log(n) - cn + \frac{n}{\log(n)}$ .
\newline
$cn\log(n) - cn + \frac{n}{\log(n)}$ $\leq$ $cn\log(n)$ (Домножим обе части неравенства на $\log(n)$)
$cn\log^{2}(n) - cn\log(n) + n$ $\leq$ $cn\log^{2}(n)$.
\newline
$n$ $\leq$ $cn\log(n)$. 
При $c$ $= 1$ начиная с $n > 1$ имеем верное неравенство, определение выполняется. %Получили верное неравенство, следовательно доказали, что $T(n)$ $\leq$ $cn\log(n)$.%

\subsection{$\Omega$}
Докажем, что $T(n) = 2T(\frac{n}{2}) + \frac{n}{\log(n)} = \Omega(n)$
\newline
А именно, что $T(n) = 2T(\frac{n}{2}) + \frac{n}{\log(n)}$ $\geq$ $cn$.
\newline
База индукции : $n = 2$, $T(2) = 2$ 
\newline
Предположение индукции : $T(n) = 2T(\frac{n}{2}) + \frac{n}{\log(n)}$ $\geq$ $cn$
%Предположим, что для всех чисел меньших $n$ это верно.
\newline%
$T(n) = 2T(\frac{n}{2}) + \frac{n}{\log(n)}$ $\geq$ $2\frac{cn}{2} + \frac{n}{\log(n)}$ $\geq$ $cn$
\newline
$cn + \frac{n}{\log(n)}$ $\geq$ $cn$. При $c = 1$ начиная с $n > 1$ имеем верное неравенство, определение выполняется.
\newline

В итоге, $T(n) = \Omega(n)$ и $T(n) = \mathcal{O}(n\log(n))$.

\section{Задача \textnumero 5}
\subsection{Функция multiply}
Функция multiply $4$ раза вызывает себя, от аргументов, длины которых в два раза меньше длин текущих аргументов. А так же, несколько раз вызывает функции, которые работают за линейное от длины аргумента время. То есть, время работы для данной функции можно задать следующей функцией(пусть $n$ в этой формуле будет максимум из длин аргументов) $T(n) = 4T(\frac{n}{2}) + n^{1}$ .

Применим Мастер-теорему. $a = 4$; $b = 2$; $c = 1$. 
$$c < \log_{b}(a) \mbox{, так как } 1 < \log_{2}(4)$$
Тогда, по Мастер-теореме, $$T(n) = \mathcal{O}(n^{\log_{b}(a)}) \mbox{, } \log_{b}(a) = \log_{2}(4) = 2 \mbox{, } T(n) = \mathcal{O}(n^{2})$$.
\subsection{Функция multiply2}
multiply2 вызывает саму себя $3$ раза, от аргументов, длины которых в два раза меньше длин текущих аргументов. А так же, несколько раз вызывает функции, которые работают за линейное от длин аргументов время. 
В данном случае, $a = 3$; $b = 2$; $c = 1$(так как функции, которые вызывает muliply2 работают за линию, их время работы $\mathcal{O}(n^{1})$)
$$c < \log_{b}(a) \mbox{, так как } 1 < \log_{2}(3)$$
Применив Мастер-теорему, получаем : 
$$T(n) = \mathcal{O}(n) = \mathcal{O}(n^{\log_{b}(a)}) \mbox{, } \log_{b}(a) = \log_{2}(3) \mbox{, } T(n) = \mathcal{O}(n^{\log_{2}(3)})$$
\end{document}