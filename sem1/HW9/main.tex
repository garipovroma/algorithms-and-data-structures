\documentclass{article}

\usepackage[utf8]{inputenc} 
\usepackage[russian]{babel}
\usepackage{amsmath,amssymb}
\usepackage{tipa}
\usepackage{hyperref}
\usepackage{graphicx}
\usepackage{comment}
\binoppenalty=10000
\relpenalty=10000
\title{Алгоритмы и Структуры Данных ДЗ-9}
\date{24.11.2019}
\usepackage{multicol}

\author{Гарипов Роман М3138}

\begin{document}
  \pagenumbering{gobble} 
  \maketitle
  \newpage
  \pagenumbering{arabic}

\section*{Задача \textnumero 1}
Посчитаем $dp_{i}$ = кол-во различных подпоследовательностей на префиксе $[0:i]$. $dp[0] = 1$, так как всего последовательностей на таком префиксе ровно 1. Рекурентная формула для этой динамики выглядит так : 
$$
dp[i - 1]=\begin{cases}
2 * dp[i - 1] + 1,\text{  если $a[i]$ не встречалось на префиксе $[0;i - 1]$;}\\
2 * dp[i - 1] - dp[x - 1] \mbox{, x = последнее вхождение $a[i]$}, \text{  иначе;}\\
\end{cases}
$$
В случае если для очередного $i$, $a[i]$ не встречалось на префиксе, мы можем просто взять любую подпоследовательность на префиксе $[0;i - 1]$ и добавить или не добавить к каждой из них $a[i]$, а так же прибавить подпоследовательность из одного элемента ${a[i]}$, поэтому формула в этом случае $dp[i] = 2 * dp[i - 1] + 1$.
\newline
В случае когда $a[i]$ уже было на префиксе, мы поступим точно так же, возьмем все подпоследовательности на префиксе и добавим или не добавим к ним $a[i]$, $dp[i] = dp[i - 1] * 2$. Но тогда мы посчитали некоторые подпоследовательности два раза. Пусть $x$ так же как и в формуле - последнее вхождение $a[i]$. Тогда понятно, что мы посчитали два раза все подпоследовательности на префиксе $[0;x - 1]$, с добавленным элементом $a[i]$ на конце, так как они были посчитаны при переходе из $dp[x - 1]$ в $dp[x]$. Вычтем их и получим верную формулу. $dp[i] = 2 * dp[i - 1] - dp[x - 1]$.
\newline
Вычисление состояние динамики по этим формулам верно, потому что мы посчитаем каждую подпоследовательность ровно один раз, как раз для этого мы вычитаем те что посчитали дважды. А так же, никакую подпоследовательность мы не упустим, потому что прибавляем $1$ в формуле для ещё не встреченного $a[i]$, добавляя все последовательности из одного элемента. Так посчитаются все различные подпоследовательности.
\newline
Таким образом, чтобы решить эту задачу, необходимо на каждой итерации цикла посчитать состояние динамики и обновить последнее вхождение последнего элемента на рассмотренном префиксе.
\newline
Ответ будет равен $dp[n - 1]$.
\section*{Задача \textnumero 2}
\section*{Задача \textnumero 3}
Будем делать примерно как в НВП за $\mathcal{O}(n\log(n))$ с бинпоиском, но немного по-другому. Вместо того чтобы уменьшать для каждой длины элемент на который заканчивается ВП этой длины, будем хранить все элементы, которые стояли в нашей динамике в этой позиции. 
\newline
Чтобы делать бинпоиск, будем использовать элемент на конце вектора, чтобы наш алгоритм работал корректно, ведь в обычном алгоритме мы просто заменяли на меньший элемент, а это и будет минимальный элемент в этом векторе, так как мы добавляем значение вектор так, что они упорядочены по убыванию(потому что находим в динамике минимальный элемент больший либо равный нашего). Поскольку изначалально каждый вектор состоит из одного элемента $inf$, для удобства будем удалять его при добавлении первого значения в соответсвующий вектор.
\newline
Теперь мы храним вектора вместо одного значения. Для каждого элемента запомним под каким номером он мог быть в НВП. 
\newline
Построим примерно таким же алгоритмом убывающую последовательность, но будем идти с конца. И запомним для каждого элемента под каким номером он может встретиться в этой убывающей последовательности.
Для каждой длины НВП будем запоминать сколько элементов могут стоять на этой позиции. Пройдемся по всем элементам векторов которые построили для первой последовательности, если номер элемента в возрастающей последовательности + номер элемента в убывающей последовательности = длине НВП - 1, тогда этот элемент точно встречается в НВП, запишем что он может стоять на своей длине в НВП. 
\newline
Пройдемся по длинам, если для текущей длины есть больше $1$ элемента претендующего встать на конец, понятно что все элементы которые могут встать на эту позицию встречаются не во всех НВП, для них запишем что они встрчаются в каких-то НВП. Если ровно $1$, то такой элемент встречается во всех НВП, для него это и запишем. 
Остальные элементы в НВП не входят.
\end{document}