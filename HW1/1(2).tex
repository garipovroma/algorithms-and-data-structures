\documentclass{article}

\usepackage[utf8]{inputenc} % если ваш файл содержит русский текст, нужно указать кодировку
\usepackage[russian]{babel} % для того, чтобы писать русский текст
\usepackage{amsmath,amssymb}
\usepackage{tipa}
\usepackage{hyperref} % для вставки ссылок
\usepackage{graphicx}
\binoppenalty=10000
\relpenalty=10000
\title{Homework 1}
\date{10.09.2019}
\author{Roman Garipov}

% до этого места была преамбула, тут можно задавать разные значения переменных, включать пакеты, а также указывать вещи, которые не суждено поменять посреди документа
\begin{document}
  \pagenumbering{gobble} % сейчас будет титульная страницу, отключим нумерацию страниц
  \maketitle % эта команда печатает титульную страницу
  \newpage % эта команда начинает новую страницу
  \pagenumbering{arabic} % включим нумерацию страниц обратно

\section{Задача \textnumero 1}
Докажите по определению, что 	$ \frac{n^{3}}{6} - 7n^{2} = \Omega(n^{3})$.
$$$$


   $ \frac{n^{3}}{6} - 7n^{2} = \Omega(n^{3})$ значит, что $ \exists $ $c > 0,  n \in \mathbb{N}: \forall$ $n_{0} > n, f(n_{0}) \geqslant cg(n_{0})$. 
   Чтобы показать, что это правда, достаточно найти такие $c$ и $n$, что выполняется определение. 
   
	Выберем $c = \frac{1}{36}$. 
   
   Тогда получим : $ \frac{n^{3}}{6} - 7n^{2} \geqslant \frac{cn^{3}}{36} $
   
    $n^{3} - 42n^{2} \geqslant \frac{cn^{3}}{6}$; 
    
    $\frac{5}{6}n^{3} - 42n^{3} \geqslant 0$;
    
     $5n^{3} - 252n^{2} \geqslant 0$; 
     
      $n^{2}(5n - 252) \geqslant 0$ 
    
     $5n - 252 \geqslant 0$; 
     
     $n \geqslant \frac{252}{5}$; $ n \geqslant 50.4$; 
     
     $n \geqslant 51$.
     
      Мы нашли такие $c$ и $n$, что определение выполняется, следовательно доказали равенство.
\end{document}